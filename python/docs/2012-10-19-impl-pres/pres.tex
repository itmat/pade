\documentclass{beamer}

\usepackage[latin1]{inputenc}
\usepackage{hyperref}
\usepackage{multirow}
\usetheme{Copenhagen}
\title[PaGE in Python]{PaGE in Python}

\institute{University of Pennsylvania}
\date{October 19, 2012}

\usepackage{listings}

\begin{document}

\begin{frame}
  \titlepage
\end{frame}

\begin{frame}{Agenda}
  \tableofcontents
\end{frame}

\section{What is PaGE?}

\begin{frame}
  \begin{centering}
  {\Large{What is PaGE?}}    
  \end{centering}
\end{frame}

\begin{frame}{What is PaGE?}
  \begin{itemize}
  \item {\bf Pa}tterns from {\bf G}ene {\bf E}xpression
  \item Discover sets of differentially expressed genes from
    microarray intensity data
  \item Written by Gregory Grant
  \item ~5,000 line Perl program
  \end{itemize}
\end{frame}

\begin{frame}{Input file}
  
  PaGE's input is a big 2-D array:
  \begin{itemize}
  \item Cells are intensities of gene expression
  \item Each row is a 'feature' (gene)
  \item Each column is a replicate
  \item Columns are grouped into conditions
  \end{itemize}
  \begin{tabular}{|l|lll|l|lll|}
    & \multicolumn{3}{|c|}{Condition 1} & ... & \multicolumn{3}{|c|}{Condition $N$}\\
    & Rep. 1 & ... & Rep. $n_1$       & ... & Rep 1 & ... & Rep. $n_{N}$\\
    \hline\\
    Gene 1 & 0.39 & ... & 4.53       & ... & 3.49 & ... & 1.06 \\
    Gene 2 & 1.33 & ... & 2.40       & ... & 0.78 & ... & 1.85 \\
    ... & ... & ... & ...       & ... & ... & ... & ... \\
    Gene M & 1.33 & ... & 2.40       & ... & 0.78 & ... & 1.85 \\
    
  \end{tabular}
  
\end{frame}

\begin{frame}{Why mess with it?}
  \begin{itemize}
  \item Originally written for microarray data
  \item In order to use for RNA-Seq, need to normalize measurements
    (since we may have a different sample size for each replicate)
  \item Current implementation doesn't handle normalization in an ideal way
  \item Improve performance for very large input files
  \end{itemize}
\end{frame}

\section{Highlights of new version}
\begin{frame}
  \begin{centering}
    {\Large{Highlights of new version}}
  \end{centering}
\end{frame}

\begin{frame}{Faster numerical operations}
  We should be able to speed it up using {\em NumPy}:
  \begin{itemize}
  \item Stands for {\bf Num}erical {\bf Py}thon something or other...
  \item Efficient storage and access to large multidimensional arrays
  \item Provides very fast numerical operations
  \item Pushes work down into in highly optimized C
  \item Lots of numerical and scientific functions
  \item Integrates well with other Python libraries
  \end{itemize}
\end{frame}

\begin{frame}{Other highlights}
  While we're at it, we can take advantage of some other libraries.
\end{frame}

\begin{frame}{Command-line parsing}
    \begin{itemize}
    \item Python's {\em argparse} library is great.
    \item Handles basic parsing of command line
    \item Converts strings to ints, floats, files, bools, etc...
    \item Does some validation
    \item Automatically generates usage information and help/error messages
    \end{itemize}
\end{frame}

\begin{frame}{Unit tests}
  \begin{itemize}
  \item Test one function at a time
  \item Writing tests first, then the code
  \item Some tests embedded in documentation
  \end{itemize}  
\end{frame}

\section{Demo}
\begin{frame}
  \begin{centering}
    \Large{Demo}        
  \end{centering}
\end{frame}


\begin{frame}{Demo}
  \begin{itemize}
  \item Run original PaGE, look at output
  \item Run new PaGE
    \begin{itemize}
    \item Show usage info
    \item Show full help message
    \item Show with bad in file
    \item Show with bad num channels
    \end{itemize}
  \item Demo iPython, basic NumPy stuff
  \item Show command-line parsing code
  \item Walk through all\_subsets
    \begin{itemize}
    \item Show docstring and doctests
    \item Use ipython to show some numpy operations
    \end{itemize}
  \item Walk through tstat in iPython
    \begin{itemize}
    \item Show vectorization
    \end{itemize}
  \item Compare timings
  \end{itemize}
\end{frame}

\end{document}
